% Introduction file, to be included in rasd.tex

\section{Introduction}
\label{sect:introduction}
\subsection{Purpose}
This document represents the RASD (Requirement Analysis and Specification Document).

The aim of this document is to completely describe the system in terms of functional and non-functional requirements, analyse the many different ways that the users may behave in order to model the system, show the constraints and the limit of the software and indicate the typical use cases that will
occur after the release.
 
This document is the baseline for project planning and estimation of size, cost, schedule. It can be also an useful starting point for the legal binding contract.

\subsection{Scope}
\subsubsection{Problem description}
CLup is a software-to-be that will help stores and customers preventing dangerous crowds during shopping due to the current Covid-19 Pandemic. This system should offer the possibility to line up from home for stores while still allowing to physically line up for stores. Moreover, it should offer the possibility to book a visit for a chosen store and get notifications about stores occupancy during the week.
The system should also allow store managers to add their stores to the system and monitor entrances.

The application has to be easy-to-use, scalable and reliable.
\subsubsection{Goals}
\begin{itemize}[itemsep=-1mm, topsep=-1mm]
	\item [\textbf{[G0]}] Allow prospective e-Customers to register to the application
	\item [\textbf{[G1]}] Allow Prospective Store Managers to register to the application by adding their store
	\item [\textbf{[G2]}] Allow e-Customers to line up from home
	\begin{itemize}[itemsep=-1mm, topsep=-1mm]\item [\textbf{[G2.1]}] Allow e-Customers to dequeue\end{itemize}
	\item [\textbf{[G3]}] Allow Store Managers to monitor entrances
	\item [\textbf{[G4]}] Allow Physical Customers to line up
	\begin{itemize}[itemsep=-1mm, topsep=-1mm]\item [\textbf{[G4.1]}] Allow Physical Customers to dequeue\end{itemize}
	\item [\textbf{[G5]}] Allow e-Customers to book a visit
	\begin{itemize}[itemsep=-1mm, topsep=-1mm]\item [\textbf{[G5.1]}] Allow e-Customers to delete a visit\end{itemize}
	\item [\textbf{[G6]}] Suggest alternative slots to balance the number of customers
	\item [\textbf{[G7]}] Allow e-Customers to manage slot notifications
\end{itemize}
% World & Machine section, to be included in introduction.tex

\subsubsection{Phenomena}
This section is about the phenomena that can occur during the life of the application. They are divide into World Phenomena and Shared Phenomena\textsuperscript{\cite{worldmachine}}. World Phenomena happen in the real world and are not visible to the application, while Shared Phenomena occur in the real world but are visible to the application. Shared Phenomena can be controlled either by the Machine (application) or by the World:

\begin{itemize}[itemsep=-1mm, topsep=-1mm]
	\item (W) means that the shared phenomena is controlled by the world
	\item (M) means that the shared phenomena is controlled by the machine
\end{itemize}

\paragraph{World Phenomena}
\begin{itemize}[itemsep=-1mm, topsep=-1mm]
	\item Going grocery shopping
	\item An e-Customer follows a path to the store
	\item A customer chooses what to buy
	\item A customer follows a path inside the store
	\item Customers stand too close near the store
	\item A wing of the store opens or closes
	\item The store makes special offers
\end{itemize}

\paragraph{Shared Phenomena}
\begin{itemize}[itemsep=-1mm, topsep=-1mm]
	\item An e-Customer is localized (M)
	\item An e-Customer gets a ticket or books a visit (W)
	\item An e-Customer gets a slot notification (M)
	\item A Customer gets added to a queue (M)
	\item A Customer visits the store for a period of time (W)
	\item An e-Customer chooses a supermarket (W)
	\item A store is at full capacity (W)
	\item A store manager changes the opening times (W)
\end{itemize}
\subsection{Definitions, Acronyms, Abbreviations}
\subsubsection{Definitions}
\begin{itemize}[itemsep=-1mm, topsep=-1mm]
	\item \textbf{Available store}: a store that can host other customers, based on closing time
	
	\item \textbf{Book a visit}: action of requesting entrance at a specific date and time (that will be guaranteed)
	
	\item \textbf{Delay window}: number of minutes a customer is allowed to enter after their turn is called (if using a ticket) or their entrance time is reached (if after a visit)
	
	\item \textbf{Get a ticket}: action of requesting the entrance as soon as possible
	
	\item \textbf{Historical data}: information about precedent store flows stored by date/time
	
	\item \textbf{Web mapping service}: a service external to the application (such as Google Maps)
	
	\item \textbf{Long term user}: for a store, an user who visited it for at least a number of times defined by the Store Managers
	
	\item Parameters:
	\begin{itemize}[itemsep=-1mm, topsep=-1mm]
		\item \textbf{M}: the dimension (in minutes) of the delay window specified by the Store Manager
		\item \textbf{P\%}: parameter specified by the Store Manager that indicates the minimum percentage of places that is reserved to tickets	
	\end{itemize}
	
	\item \textbf{Reasonably available store}: a store for which waiting time is less than a user defined parameter
	
	\item \textbf{Spatial locality}: distance from the user inferior to a user defined parameter 
	
	\item \textbf{Store information}: refers to store id, address, customer capacity, opening times, allowed delay of a customer, items/categories 
	
	\item \textbf{User}: an user of the application. As per the user diagram, the term encompasses both e-Customers and Store Managers	
\end{itemize}

\subsubsection{Acronyms and Abbreviations}
\begin{itemize}[itemsep=-1mm, topsep=-1mm]
	\item \textbf{eC}: e-Customer
	\item \textbf{SM}: Store Manager
	\item \textbf{RASD}: Requirement Analysis and Specification Document
	\item \textbf{[Gn]}: n-th goal
	\item \textbf{[Rn]}: n-th requirement
	\item \textbf{[Dn]}: n-th domain assumption
	\item Internet Protocols:
	\begin{itemize}[itemsep=-1mm, topsep=-1mm]
		\item \textbf{HTTPS}: Hypertext Transfer Protocol Secure 
		\item \textbf{TLS}: Transport Layer Security
		\item \textbf{REST}: REpresentational State Transfer
	\end{itemize}
\end{itemize}	

\subsection{Revision history}
\begin{center}
	\begin{tabular}{c | c | c}	
		Version & Date & Comment \\ \hline
		1.0 & 2020-12-22 & First release
	\end{tabular}
\end{center}


\subsection{Reference documents}
R\&DD Assignment AY 2020-2021, \textit{Software Engineering 2's BeeP page}

\subsection{Document Structure}
This document is composed of the following parts:
\begin{itemize}[itemsep=-1mm, topsep=-1mm]
	\item \textbf{Section \ref{sect:introduction}} is an introduction to the document and offers an overview of the project, describing what its goals are and giving pointers to the following sections
	\item \textbf{Section \ref{sect:overview}} gives an overview of the project, showing its usages and its domain. Actors are also stated in this section, alongside state and class diagrams
	\item \textbf{Section \ref{sect:requirements}} identifies the system's requirements, both functional and non-functional, then presents the application's use cases and its sequence diagrams
	\item \textbf{Section \ref{sect:alloy}} contains an Alloy model of the system with proofs of consistency
	\item \textbf{Section \ref{sect:effort}} reports the hours spent by this group's components and the tool used during the development of this document 
\end{itemize}