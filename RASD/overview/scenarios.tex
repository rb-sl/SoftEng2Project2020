% Scenarios file, to be included in overview.tex

\subsubsection{Scenarios}
The following scenarios are provided to describe possible situations during the usage of CLup.

\paragraph{Scenario 1}
Nathan, during the pandemic, needs a way to safely go grocery shopping, avoiding long queues out of supermarkets. His friend John suggests him to download the CLup app to line up from home; Nathan is intrigued and immediately downloads it, signs up and enqueues for his favorite shop.

\paragraph{Scenario 2}
After being enqueued for about thirty minutes, Nathan receives a notification telling him his turn is close and he needs to leave home. After reaching the supermarket he waits for some minutes until his number gets displayed by the shop; he then passes the QR code shown by the application at a scanner and enters the shop. Once he has done, he scans again the code at the exit and goes home. 

\paragraph{Scenario 3}
Ms. Margareth is not accustomed to technology, so she does not use a smartphone; as she needs to go grocery shopping, she goes directly to the store and gets a physical ticket, so she can wait outside with few people, as most use CLup. Once she sees her number is reached, she scans the ticket and enters.

\paragraph{Scenario 4}
Nathan knows Julie, the owner of a bakery close to his house, so he tells her about CLup. She is curious and downloads the app, registering her shop. She then logs in as Store Manager into the tablet she uses for work, so that her employees can use it to scan codes and monitor the entrance of customers.

\paragraph{Scenario 5}
Anthony works for a supermarket that adopted CLup. As a store manager, he has the duty of keeping updated the store’s inventory and make customers respect their turns. Luckily, the store decided to install some QR code readers connected to CLup at entrances, so he doesn’t have to scan every customer’s code himself. 

\paragraph{Scenario 6}
As Joe had some free time, he decided to get a physical ticket for the supermarket, hoping few people were already lined up. After 10 minutes of being enqueued he receives an important call and needs to leave; he decides to scan his ticket to dequeue and let the people behind him enter some time before.

\paragraph{Scenario 7}
Andrew is at home, waiting for his turn after he requested a ticket. Suddenly he receives a notification from CLup, letting him know that someone in front of him dequeued and giving him the possibility to enter earlier than expected. He gladly accepts and gets ready to go.

\paragraph{Scenario 8}
Anastasia knows she needs to go shopping for groceries the next day but has a very tight schedule. Luckily, she can use CLup to book a visit for the exact time she chooses, so she requests a reservation. The next day she can enter without having to line up, thus saving much of her time.

\paragraph{Scenario 9}
Confident about his free time, Nick booked three visits for the next week. The day before the second one he remembered he was actually busy, so he logged in the application and canceled the reservation.